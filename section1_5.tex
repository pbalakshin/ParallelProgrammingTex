{ %section1_5
	\subsection{Automatic parallelization of programs}
	\par Parallel computing is a rather complicated manual process, so it seems obvious that it needs to be automated using a compiler. Such attempts are made, however, the efficiency of auto-parallelization is not yet sufficient, because good indicators of parallel acceleration are achieved only for a limited set of simple for-cycles in which there are no data dependencies between iterations and the number of iterations cannot change after the start of the cycle. But even if the two indicated conditions are satisfied in a certain for-cycle, but it has a complex non-obvious structure, then its parallelization will not be performed. Types of automatic parallelization:
	\begin{itemize}
		\item\textit{Fully automatic:}\quad participation of the programmer is not required, all actions are performed by the compiler.
		\item\textit{Semi-automatic:}\quad programmer gives instructions to the compiler in the form of special keys that allow you to adjust some aspects of parallelization.
	\end{itemize}
	\par Weaknesses of automatic parallelization:
	\begin{itemize}
		\item erroneous change in program logic;
		\item speed reduction instead of increase;
		\item lack of manual parallelization flexibility;
		\item only cycles are efficiently parallelized;
		\item inability to parallelize programs with a complex algorithm of work.
	\end{itemize}
	\par Here are examples of how the c-program in the src.c file can be automatically parallelized using some popular compilers:
	\begin{itemize}
		\item Compiler GNU Compiler Collection:	 \\
\texttt{gcc -O3 -floop-parallelize-all \\-ftree-parallelize-loops=K \\-fdump-tree-parloops-details src.c}. \\ In this case, the programmer can choose the value of the parameter K, which is recommended to be set equal to the number of cores (processors). Features of the auto-parallelization implementation in gcc are dedicated to an independent project:\\ \url{https://gcc.gnu.org/wiki/AutoParInGCC}. 
		\item Intel compiler:	 
\texttt{icc -c -parallel -par-report file.cc}
		\item Oracle compiler:	 
\texttt{solarisstudio -cc -O3 -xautopar \\-xloopinfo src.c}
	\end{itemize}
}
