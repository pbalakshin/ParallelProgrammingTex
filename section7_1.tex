{ %section7_1
	\subsection{Work sequence}
	\begin{enumerate}
		\item1.	In the program resulting from the execution of Laboratory research~\#3, change the Generate stage so that the generated set of random numbers does not depend on the number of threads running the program. For example, before calling \texttt{rand\textunderscore r},  at each iteration of $i$, you can call the \texttt{srand(f(i))} function, where \texttt{f} is an arbitrary function. You can think of and use any other method.
		\item Replace the gettimeofday function calls to \texttt{omp\textunderscore get\textunderscore wtime}.
		\item Parallelize the calculations at the Sort stage, for which you can perform sorting in two stages:
			\begin{itemize}
				\item Sort the first and the second half of the array in two independent threads (you can use the OpenMP Directive ''parallel sections''); 
				\item Combine the sorted halves into a single array.
			\end{itemize}
		\item Write a function that outputs a message to the console about the current percentage of program shutdown once a second. Run the specified function in a separate thread running in parallel with the main computing cycle.
		\item Ensure forward compatibility of a written parallel program. To do this, conditionally redefine all called functions of the form «\texttt{omp\textunderscore *}» in preprocessor directives, for example:
			\begin{figure}[H]
				\lstinputlisting{lab4Example.cpp}
			\end{figure}
		\item Conduct the experiments with the resulting program, varying $N$ from $min(\frac{N_x}{2},\;N_1)$ to $N_2$, for the same values $N_1$ and $N_2$, which were used in laboratory research \#1, where $N_x$ is the number of $N$, in which the overhead of parallelization exceeds the gain from parallelization. Write a report on the work performed. Be ready to answer questions on the presentation.
		\item\textbf{Optional task (to get good and excellent mark).} Reduce the number of iterations of the main loop from 100 to 10 and perform experiments by measuring the execution time using the following methods:
			\begin{itemize}
				\item Use the minimum measurement of ten received ones; 
				\item Calculate the confidence interval with a confidence level of 95\% based on ten dimensions.
			\end{itemize}
			Give graphs of parallel acceleration for both methods in the same coordinate system, furthermore, giving the lower and upper bounds of the confidence interval by two independent graphs is advisable.
		\item\textbf{Optional task (to get highest mark):} in part 3 at the Sort stage, perform parallel sorting of $k$ parts of the array (not two parts) in $k$ threads, where $k$ is the number of processors (cores) in the system, which becomes known only at the program execution stage using the command \texttt{«k = omp\textunderscore get\textunderscore num\textunderscore procs()}».
	\end{enumerate}
}