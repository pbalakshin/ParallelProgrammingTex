{ %section1_7
	\subsection{Atomic operations in a multithreaded program}
	\label{atomic:section}
	\par The main problem with parallel computing is the need to resolve conflicts while accessing the shared memory of several threads. To solve this problem, they usually try to streamline the access of streams to shared data using special tools - synchronization primitives. However, the question arises whether there are such atomic atomic operations, the execution of which by several threads simultaneously does not require synchronization of actions, because these operations would be performed by the processor in one fell swoop, or, as they say, atomically (i.e., no other operation can push the previous atomic operation out of the processor before it is completed).
	\par Almost all assembler instructions are such operations, since at a low level they use only those operations that are present in the processor command system, which means that they can be performed atomically (uninterruptedly). However, when compiling a C program, C language commands are usually translated into several assembler instructions. In this regard, the question arises about the possible existence of C-commands that are compiled into one assembler instruction. Such commands could not be "protected"\ by synchronization primi-tives (mutexes) in parallel computing.
	\par However, there are very few such operations, and some of them can behave both atomically and non-atomically depending on the hardware platform for which the C program is compiled. Consider the simplest increment command of an integer variable (int type) in C: \texttt{"w++"}. You can easily verify (for example, using the gcc compiler's \texttt{"\--S"}\ key) that this command will be translated into three assembler instructions (take from memory, increase, put back):
	\begin{figure}[H]
		\lstinputlisting[language={[x86masm]Assembler}]{atomicOperationExample1.asm}
	\end{figure}
	\par This means that it is not safe to perform the increment operation of a variable in several threads at the same time, because when executing assembler instruction 2, the thread can be interrupted and the processor transferred to another thread, which will receive an incorrect value of the undereduced variable.
	\par It is logical to assume that assignment operations should not have the described problem. Indeed, in Assembler there is a separate instruction for writing the value of a variable to the specified address. Unfortunately, this assumption is not completely true: when assigning a variable of type char, this operation will be performed by a single assembler instruction. However, with other data types this cannot be said for sure. The general rule of thumb can be roughly stated as follows: "the atomicity of the assignment operation is guaranteed only for operations with data whose bit capacity does not exceed the processor bit capacity."
	\par For example, when assigning an int variable to a 32-bit processor, one assembler instruction will be generated. However, when compiling the same operation on a 16-bit computer, two assembler instructions will be generated to write the low and high bits independently.
	\par The formulated rule works with the assignment of variables and expressions, however, it cannot always be satisfied with the assignment of constants. Consider an example of C-code in which a 64-bit variable s (type uint64\textunderscore t) is assigned a large number, obviously exceeding the 32-bit value:
	\begin{figure}[H]
		\lstinputlisting{atomicOperationExample2.cpp}
	\end{figure}
	\par This code will be translated into the following assembler code on a 64-bit processor:
	\begin{figure}[H]
		\lstinputlisting[language={[x86masm]Assembler}]{atomicOperationExample3.asm}
	\end{figure}
	\par As you can see, the assignment operation was translated into two assemb-ler instructions, which makes it impossible to safely parallelize such an operation.
	\par This rule applies not only to the assignment operation, but also to the operation of reading a variable from memory, so any of these operations in a thread-safe environment will have to be protected by mutexes or critical sections.
	\par A special case of atomic data changes is structure changes. To do this, we need to use a CAS operation with a pointer to this structure. Performing such an operation, the processor will create a second data structure with the specified fields and compare it with the old version of the structure. If the value of at least one field has changed, then it will atomically replace the pointer. There is overhead in this: even a simple change of one field of the structure requires the creation of a full copy of the structure, then to replace the pointer.
	\par
}